\documentclass[12pt,a4paper]{article}

% for polish language
\usepackage{polski}

% for some math symbols
\usepackage{amssymb}

% correct footnotes placement
\usepackage[bottom]{footmisc}

% for \say command
\usepackage{dirtytalk}

% change title of the bibliography
\def\bibname{Referencje}\let\refname\bibname

% links
\usepackage{hyperref}
\hypersetup{
    colorlinks,
    citecolor=black,
    filecolor=black,
    linkcolor=black,
    urlcolor=black
}

% image displaying
\usepackage{subcaption}
\usepackage{graphicx}

\title{Dokumentacja do projektu z ALHE}

\author{Rafal Babinski \and Roman Moskalenko}
\date{}

\begin{document}

\maketitle

\section{Treść zadania}

\textbf{SK.ALHE.4}\\

Złodziej ukradł $X$ gramów złota ze skarbca i wraca do domu pociągiem. Żeby uniknąć schwytania przez policję, musi zamienić złoto na banknoty, więc postanawia sprzedać złoto pasażerom pociągu. 
Zainteresowanych kupnem jest $N$ pasażerów, każdy z nich zgadza się kupić $a_{i}$, $i \in (1,2, ..., N)$ gramów złota za $v_{i}$, $i \in (1,2, ..., N)$. Złodziej chce uciec przed policją, jednocześnie maksymalizując zysk. 
Zaimplementuj program bazujący na  algorytmie ewolucyjnym, który wskaże pasażerów, którym powinien sprzedać złoto oraz sumę wartości banknotów, którą zarobi. Zastosowanie i porównanie  z innym algorytmem będzie dodatkowym atutem przy ocenie projektu.

\section{Projekt wstępny}

\subsection{Założenia ogólne}

Ewidentnie mamy do czynienia z problemem plecakowym zero-jedynkowym. Zakładamy, że $a_{i}, v_{i}, X \in \mathbb{R^+}$.

W projekcie będziemy nawiązywać się do artykułu\cite{knapsack}, w którym badano efektywność algorytmów genetycznych dla problemu plecakowego zero-jedynkowego. Dalej będziemy odnosić się do niego po prostu jako \say{artykuł}.

Preferowanym językiem programowania jest Python.

\pagebreak


\subsection{Planowane rozwiązanie}
W ramach projektu planujemy porównać działanie trzech algorytmów:

\begin{itemize}

  \item \textbf{Prosty algorytm genetyczny} -- algorytm genetyczny, w którym odrzuca się rozwiązania niespełniające ograniczenia problemu.

  \item \textbf{Algorytm genetyczny zmodyfikowany} -- algorytm genetyczny bazowany na artykule. 
Polega on na poradzeniu sobie z rozwiązaniami niespełniającymi ograniczenia.
Do tego zaimplementujemy dwie metody, które najlepiej się sprawdziły wg. wspomnianego artykułu: \emph{zastosowanie w funcji celu logarytmicznej funkcji kary} oraz \emph{zachłanne naprawianie rozwiązań} (do wyboru odpowiednio do pojemności plecaka).

  \item \textbf{Branch-and-Bound} -- algorytm, który powinien zapewnić nam rozwiązanie optymalne. Znając rozwiązanie optymalne będziemy w staniedokładniej oszacować efektywność algorytmów genetycznych.
  
\end{itemize}

Oczekujemy, że zmodyfikowana wersja algorytmu genetycznego okaże się efektywniejsza niż wersja prosta.

\section{Realizacja projektu}

\subsection{Generowane danych testowych}

Na potrzeby projektu generowano instancje problemu plecakowego w sposób opisany w artykule.

\subsubsection{Korelacja}

Rozróżnia się instancje, gdzie wagi i  zyski są: 
\begin{itemize}
  \item nieskorelowane -- współczynnik korelacji w przybliżeniu równy 0,
  \item skorelowane słabo -- współczynnik korelacji w przedziale (0, 1),
  \item skorelowane mocno -- współczynnik korelacji równy 1.
\end{itemize}

Większa korelacja, jak wspomniano w artykule, implikuje większą trudność problemu.

Duży wpływ na współczynnik korelacji mają parametry $v$ -- wartość maksymalna wag oraz $r$ -- odchylenie zysku od wagi. Nie będziemy szczególnie badać tej zależności. W dalszych rozdziałach wykorzystane dane zostały wygenerowane z parametrami $v=10$, $r=5$. 

\begin{figure}
  \begin{subfigure}[t]{.475\textwidth}
    \centering
    \includegraphics[width=\linewidth]{img/data_none002.png}
    \caption{}
  \end{subfigure}
  \hfill
  \begin{subfigure}[t]{.475\textwidth}
    \centering
    \includegraphics[width=\linewidth]{img/corrcoefnone.png}
    \caption{}
  \end{subfigure}
  
  \medskip
  
  \begin{subfigure}[t]{.475\textwidth}
    \centering
    \includegraphics[width=\linewidth]{img/data_weak062.png}
    \caption{}
  \end{subfigure}
  \hfill
  \begin{subfigure}[t]{.475\textwidth}
    \centering
    \includegraphics[width=\linewidth]{img/corrcoefweak061.png}
    \caption{}
  \end{subfigure}
  
   \begin{subfigure}[t]{.475\textwidth}
    \centering
    \includegraphics[width=\linewidth]{img/data_strong.png}
    \caption{}
  \end{subfigure}
  \hfill
  \begin{subfigure}[t]{.475\textwidth}
    \centering
    \includegraphics[width=\linewidth]{img/corrcoefstrong.png}
    \caption{}
  \end{subfigure}

  \caption{1000 próbek danych (przy $v=10, r=5$) nieskorelowanych (a), skorelowanych słabo (c) i mocno skorelowanych (e). Współczynnik korelacji w 1000 wygenerowanych instancji bez korelacji (b), z korelacją słabą (d) i mocną (f).}
\end{figure}

\pagebreak

Zwrócimy uwagę, że dane skorelowane \say{słabo} mają współczynnik korelacji w przybliżeniu równy 0.61 co możemy uznać za słabą korelację.

\subsubsection{Pojemność plecaka}

W artykule opisano 2 typy pojemności:
\begin{itemize}
  \item Ograniczona -- $2v$, gdzie $v$ jest wartością maksymalną wag.
  \item Średnia  -- równa połowie sumy wszystkich wag.
\end{itemize}

Nasz generator umożliwia również ustawienie pojemności wybranej arbitralnie przez użytkownika.

\subsection{Branch-and-Bound}


%\subsection{Dane wejściowe}

%Na potrzeby eksperymentów dane wejśiowe będą generowane losowo.
%Wprowadzimy słabą korelację (tak jak opisano w artykule) między wagami a zyskami:

%\medskip
%$W[i] := (uniformly)$ random$([1..v])$

%$P[i] := W[i] + (uniformly)$ random$([-r..r])$, 
%\medskip

%gdzie \textbf{$W$} i \textbf{$P$} są wektorami wag oraz zysków odpowiednio, $v, r \in \mathbb{R^+}$.

%Jeśli, dla pewnych $i$, $P[i]\leq 0$, taka wartość zysku zostanie zignorowana i wylosowana ponownie, aż $P[i] > 0$.
%\medskip

%Także będzie możliwość podania przez użytkownika danych wejściowych jako plik.

\pagebreak

\begin{thebibliography}{9}
\bibitem{knapsack} 
Zbigniew Michalewicz, Jarosław Arabas. "Genetic Algorithms for the 0/1 Knapsack Problem." 1994.
\end{thebibliography}

  
\end{document}
