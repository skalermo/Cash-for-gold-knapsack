\documentclass[12pt,a4paper]{article}

% for polish language
\usepackage{polski}

% for some math symbols
\usepackage{amssymb}

% correct footnotes placement
\usepackage[bottom]{footmisc}

% for \say command
\usepackage{dirtytalk}

% change title of the bibliography
\def\bibname{Referencje}\let\refname\bibname

% for advanced math formulas
\usepackage{mathtools}

% links
\usepackage{hyperref}
\hypersetup{
    colorlinks,
    citecolor=black,
    filecolor=black,
    linkcolor=black,
    urlcolor=black
}

% image displaying
\usepackage{subcaption}
\usepackage{graphicx}

\usepackage{multirow} 
\usepackage{makecell}

\title{Dokumentacja do projektu z ALHE}

\author{Rafal Babinski \and Roman Moskalenko}
\date{}

\begin{document}

\maketitle

\section{Treść zadania}

\textbf{SK.ALHE.4}\\

Złodziej ukradł $X$ gramów złota ze skarbca i wraca do domu pociągiem. Żeby uniknąć schwytania przez policję, musi zamienić złoto na banknoty, więc postanawia sprzedać złoto pasażerom pociągu. 
Zainteresowanych kupnem jest $N$ pasażerów, każdy z nich zgadza się kupić $a_{i}$, $i \in (1,2, ..., N)$ gramów złota za $v_{i}$, $i \in (1,2, ..., N)$. Złodziej chce uciec przed policją, jednocześnie maksymalizując zysk. 
Zaimplementuj program bazujący na  algorytmie ewolucyjnym, który wskaże pasażerów, którym powinien sprzedać złoto oraz sumę wartości banknotów, którą zarobi. Zastosowanie i porównanie  z innym algorytmem będzie dodatkowym atutem przy ocenie projektu.

\section{Projekt wstępny}

\subsection{Założenia ogólne}

Ewidentnie mamy do czynienia z problemem plecakowym zero-jedynkowym. Zakładamy, że $a_{i}, v_{i}, X \in \mathbb{R^+}$.

W projekcie będziemy nawiązywać się do artykułu\cite{knapsack}, w którym badano efektywność algorytmów genetycznych dla problemu plecakowego zero-jedynkowego. Dalej będziemy odnosić się do niego po prostu jako \say{artykuł}.

Preferowanym językiem programowania jest Python.

\pagebreak


\subsection{Planowane rozwiązanie}
W ramach projektu planujemy porównać działanie trzech algorytmów:

\begin{itemize}

  \item \textbf{Prosty algorytm genetyczny} -- algorytm genetyczny, w którym odrzuca się rozwiązania niespełniające ograniczenia problemu.

  \item \textbf{Algorytm genetyczny zmodyfikowany} -- algorytm genetyczny bazowany na artykule. 
Polega on na poradzeniu sobie z rozwiązaniami niespełniającymi ograniczenia.
Do tego zaimplementujemy dwie metody, które najlepiej się sprawdziły wg. wspomnianego artykułu: \emph{zastosowanie w funcji celu logarytmicznej funkcji kary} oraz \emph{zachłanne naprawianie rozwiązań} (do wyboru odpowiednio do pojemności plecaka).

  \item \textbf{Branch-and-Bound} -- algorytm, który powinien zapewnić nam rozwiązanie optymalne. Znając rozwiązanie optymalne będziemy w staniedokładniej oszacować efektywność algorytmów genetycznych.
  
\end{itemize}

Oczekujemy, że zmodyfikowana wersja algorytmu genetycznego okaże się efektywniejsza niż wersja prosta.

\section{Realizacja projektu}

\subsection{Generowane danych testowych}

Na potrzeby projektu generowano instancje problemu plecakowego w sposób opisany w artykule.

\subsubsection{Korelacja}

Rozróżnia się instancje, gdzie wagi i  zyski są: 
\begin{itemize}
  \item nieskorelowane -- współczynnik korelacji w przybliżeniu równy 0,
  \item skorelowane słabo -- współczynnik korelacji w przedziale (0, 1),
  \item skorelowane mocno -- współczynnik korelacji równy 1.
\end{itemize}

Większa korelacja, jak wspomniano w artykule, implikuje większą trudność problemu.

Duży wpływ na współczynnik korelacji mają parametry $v$ -- wartość maksymalna wag oraz $r$ -- odchylenie zysku od wagi. Nie będziemy szczególnie badać tej zależności. W dalszych rozdziałach wykorzystane dane zostały wygenerowane z parametrami $v=10$, $r=5$. 

\begin{figure}
  \begin{subfigure}[t]{.475\textwidth}
    \centering
    \includegraphics[width=\linewidth]{img/data_none002.png}
    \caption{}
  \end{subfigure}
  \hfill
  \begin{subfigure}[t]{.475\textwidth}
    \centering
    \includegraphics[width=\linewidth]{img/corrcoefnone.png}
    \caption{}
  \end{subfigure}
  
  \medskip
  
  \begin{subfigure}[t]{.475\textwidth}
    \centering
    \includegraphics[width=\linewidth]{img/data_weak062.png}
    \caption{}
  \end{subfigure}
  \hfill
  \begin{subfigure}[t]{.475\textwidth}
    \centering
    \includegraphics[width=\linewidth]{img/corrcoefweak061.png}
    \caption{}
  \end{subfigure}
  
   \begin{subfigure}[t]{.475\textwidth}
    \centering
    \includegraphics[width=\linewidth]{img/data_strong.png}
    \caption{}
  \end{subfigure}
  \hfill
  \begin{subfigure}[t]{.475\textwidth}
    \centering
    \includegraphics[width=\linewidth]{img/corrcoefstrong.png}
    \caption{}
  \end{subfigure}

  \caption{1000 próbek danych (przy $v=10, r=5$) nieskorelowanych (a), skorelowanych słabo (c) i mocno skorelowanych (e). Współczynnik korelacji w 1000 wygenerowanych instancji bez korelacji (b), z korelacją słabą (d) i mocną (f).}
\end{figure}

\pagebreak

Zwrócimy uwagę, że dane skorelowane \say{słabo} mają współczynnik korelacji w przybliżeniu równy 0.61 co możemy uznać za słabą korelację.

\subsubsection{Pojemność plecaka}

W artykule opisano 2 typy pojemności:
\begin{itemize}
  \item Ograniczona -- $2v$, gdzie $v$ jest wartością maksymalną wag.
  \item Średnia  -- równa połowie sumy wszystkich wag.
\end{itemize}

Nasz generator umożliwia również ustawienie pojemności wybranej arbitralnie przez użytkownika.

\

Dla wygody posługiwania dodaliśmy też listę indeksów do elementów posortowanej wg. stosunku zysk/waga tablicy. Nie wpływa to w żaden sposób na rozwój osobników w naszych algorytmach, w dodatek przyspiesza naprawianie zachłanne.

\subsection{Branch-and-Bound}

Metoda przeglądu drzewa przestrzeni rozwiązań z heurystycznym odcinaniem gałęzi, które nie mają rozwiązania optymalnego.
Nasza implementacja wykorzystuje przegląd drzewa wgłąb.

\subsubsection{Wykorzystanie zasobów czasowych i pamięciowych}

Przedstawimy kilka wykresów opisujących zależność czasu trwania algorytmu od liczby elementów w plecaku przy różnych korelacji i pojemności.

\medskip

Rysunek 2 dobrze opisuje jak bardzo czas trwania algorytmu zależy od korelacji danych i typu pojemności plecaka. Dla każdego punktu na wykresie algorytm został odpalony 5 razy i został zachowany czas uśredniony. 

\medskip

Jak widać dla \textbf{słabej korelacji} i ograniczonej pojemności algorytm działa bardzo szybko (mniej niż 1 sekunda). Tak się dzieje dla tego, że ograniczona pojemność pozwala algorytmowi dość często odrzucać gałęzie drzewa. 

\medskip

Dla \textbf{silnej korelacji} czas trwania jest rzędu dziesiątek sekund, czasami sięga kilku minut.

\medskip

Dla \textbf{powiększonej pojemności} plecaka wykres czasu trwania zaczyna przypominać funkcję wykładniczą. Także zauważyliśmy wykorzystanie przez algorytm zbyt dużej ilości pamięci (czasami system operacyjny zabijał program) dla tego postanowiliśmy nie kontynuować pomiaru czasu dla większych wartości.

\begin{figure}[h]
  \centering
\begin{subfigure}{0.5\textwidth}
  \includegraphics[width=\linewidth]{img/bnb_weakcorr_restrcap_500.png}
  \caption{}
  \label{}
\end{subfigure}\hfil
\begin{subfigure}{0.5\textwidth}
  \includegraphics[width=\linewidth]{img/bnb_strongcorr_restrcap_350.png}
  \caption{}
  \label{}
\end{subfigure}

\medskip
\begin{subfigure}{0.5\textwidth}
  \includegraphics[width=\linewidth]{img/bnb_weakcorr_avgcap_45.png}
  \caption{}
  \label{}
\end{subfigure}

\caption{Wykresy zależności czasu działania algorytmu od liczby elementów w plecaku przy różnych właściwościach zbioru: (a) słaba korelacja i ograniczona pojemność plecaka; (b) silna korelacja i ograniczona pojemność plecaka; (c) słaba korelacja i rozszerzona pojemność.}

\end{figure}

\subsubsection{Wykorzystanie algorytmu w projekcie}

Czas trwania i pochłanianie pamięci robią algorytm bardzo nieatrakcyjny. Spróbowaliśmy nawet zmodyfikować algorytm tak, aby nie zagłębiał się niżej określonego poziomu. To rzeczywiście przyspieszyło jego działanie, ale zwracane przez algorytm wartości stały się bardzo odległe od wartości optymalnej, co robi modyfikację niezbyt efektywną. Z tego powodu postanowiliśmy zrezygnować z użycia tego algorytmu w przypadku rozszerzonej pojemności plecaka.
\pagebreak

\subsection{Algorytm genetyczny}

Wyróżniamy następujące komponenty zaimplementowanych algorytmów genetycznych (różnice między algorytmami opiszemy w odpowiednich podrozdziałach każdego z komponentów):

\begin{enumerate}

\item Stworzenie populacji początkowej

\item Selekcja osobników do krzyżowania

\item Krzyżowanie

\item Mutacja

\item Selekcja osobników do utworzenia następnej populacji

\end{enumerate}

Każdy z tych komponentów postaramy się szczegółowo opisać.

\subsubsection{Populacja początkowa}

Inicjalizacja populacji początkowej odbywa się w ten sposób, że dla każdego nowego osobnika losowana jest lista zer i jedynek.

Na koniec dla każdego osobnika populacji wyliczana jest jakość (fitness).
W przypadku algorytmu z funkcją kary, osobniki nie są naprawiane, natomiast fitness jakość uwzględnia karę. Dla pozostałych algorytmów osobniki przy przekroczeniu pojemności plecaka zostają naprawione naprawianiem losowym.

Po tych czynnościach populacja jest gotowa do wejścia do pętli głównej algorytmu.

\subsubsection{Selekcja do krzyżowania}

Zgodnie z artykułem zaimplentowaliśmy selekcję turniejową. W każdym turnieju uczęstniczy 2 osobników. Ponieważ przy utworzeniu nowej populacji osobniki zostają posortowane wg. jakości, wystarczy porównać ich indeksy aby wyznaczyć zwyciężcę turnieju.

\subsubsection{Krzyżowanie}

Zaimplementowaliśmy krzyżowanie jednopunktowe, używane w artykule.
Dodaliśmy też krzyżowanie wymiejające: losuje się $k$ indeksów, gdzie $k \in (1.. \left\lfloor \dfrac{N}{2} \right\rfloor)$. Dla każdego indeksu robimy zamianę genów osobników.

W wyniku tej operacji powstają dwa nowe osobniki z przeliczonymi wartościami jakości.

Współczynnik krzyżowania odpowiada za to jak często stosowane jest krzyżowanie: np. dla współczynnika równego 1 krzyżujemy zawsze, dla 0.5 -- w przybliżeniu w połowie przypadków, dla 0 -- nigdy nie krzyżujemy.

\subsubsection{Mutacja}

Odwrócenie wybranego bitu.

Każdy bit dowolnego osobnika może zostać zmutowany, niezależnie od tego, czy dany osobnik został wybrany do krzyżowania, czy nie.

Współczynnik mutacji odpowiada za to, jak często mutujemy bity.

\subsubsection{Selekcja do następnej populacji}

Wszystkie osobniki zostają posortowane malejąco wg. posiadanej jakości. W przypadku algorytmu bez modyfikacji osobniki przekraczające pojemność plecaka zostają zastąpione wylosowanymi na nowo, które już spełniają ten warunek.

\subsection{Porównanie algorytmów}

W tym rozdziale porównamy działanie algorytmów. 

Do porównania: algorytm genetyczny zwykły, bez modyfikacji (\emph{vanilla}), algorytm genetyczny z logarytmiczną funkcją kary (\emph{penalty}), algorytm genetyczny z zachłannym naprawianiem osobników (\emph{repair}).

Wydajność algorytmu oceniamy na podstawie najlepszej uzyskanej wartości jakości w ciągu 500 generacji. Współczynniki krzyżowania i mutacji są stałe i wynoszą 0.65 i 0.05 odpowiednio. Różnica między naszym eksperymentem a przedstawionym w artykule jest taka, że dla rozszerzonej pojemności plecaka nie jesteśmy w stanie sprawdzić czy znalezione rozwiązania jest optymalne. Także dla algorytmu z funkcją kary po zakończeniu algorytmu rozwiązania często przekraczają pojemność plecaka, dla tego jednorazowo naprawiamy je zachłannie. W tabeli poniżej przedstawiamy uzyskane rezultaty (dokładniejszy przebieg eksperymentu można znaleźć w logach projektu).

\pagebreak

\begin{table}[]
\begin{tabular}{|c|c|c|c|c|c|}
\hline
\multirow{2}{*}{Correl.} &
  \multirow{2}{*}{\begin{tabular}[c]{@{}c@{}}No. of\\ items\end{tabular}} &
  \multirow{2}{*}{\begin{tabular}[c]{@{}c@{}}Cap.\\ type\end{tabular}} &
  \multicolumn{3}{c|}{Algorithm} \\ \cline{4-6} 
       &     &    & vanilla & repair & penalty \\ \hline
none   & 100 & c1 & 49.5 (89.9) & 89.9 (89.9) & 87.0 (89.9)\\
       &     & c2 & 349.9 (-) & 389.4 (-) & 390.4 (-) \\
       & 250 & c1 & 56.3 (105.0) & 105.0 (105.0) & 102.2 (105.0) \\
       &     & c2 & 846.0 (-) & 959.5 (-) & 954.7 (-) \\
       & 500 & c1 & 35.8 (124.9) & 124.9 (124.9) & 120.6 (124.9) \\
       &     & c2 & 1573.3 (-) & 1726.1 (-) & 1735.4 (-)\\ \hline
weak   & 100 & c1 & 36.9 (54.5) & 54.5 (54.5) & 54.5 (54.5) \\
       &     & c2 & 362.5 (-) & 394.5 (-) & 398.9 (-) \\
       & 250 & c1 & 35.8 (63.6) & 63.6 (63.6) & 61.2 (63.6) \\
       &     & c2 & 826.5 (-) & 934.3 (-) & 934.5 (-) \\
       & 500 & c1 & 33.0 (79.8) &  (79.8)       &         \\
       &     & c2 &         &        &         \\ \hline
strong & 100 & c1 &         &        &         \\
       &     & c2 &         &        &         \\
       & 250 & c1 &         &        &         \\
       &     & c2 &         &        &         \\
       & 500 & c1 &         &        &         \\
       &     & c2 &         &        &         \\ \hline
\end{tabular}
\end{table}


\subsection{Opis funkcjonalny}

Projekt składa się z kilku plików .py, które zawierają najistotniejsze komponenty projektu:

\begin{itemize}

\item \textbf{BNBAlgorithm.py} - implementacja algorytmu Branch-and-Bound

\item \textbf{Chromosome.py} - plik zawierający klasę \emph{Chromosome}, oraz dotyczące jej metody. W tym pliku znajdują się implementacje funkcji do naprawiania osobników, mutacji, krzyżowania oraz wyliczenia jakości z uwzględnieniem kary.

\item \textbf{DrawUnit.py} - plik zawierający metody do wizualizacji populacji osobników. Zrezygnowano z korzystania z nich, ponieważ dość trudno interpretować powstające obrazki.

\item \textbf{Experiments.py} - w pliku algorytmy zostają uruchomione dla różnych zestawów danych wejściowych. Wszystkie wyniki są zapisywane w folderze \emph{logs/}

\item \textbf{Generator.py} - generowanie danych syntetycznych, które przekazywane są badanym algorytmom. Można ustawiać korelacje generowanych danych, zakres danych, pojemność bądź typ pojemności plecaka, liczbę elementów w plecaku oraz ziarno losowości.

\item \textbf{GeneticAlgorithm.py} - metody algorytmu genetycznego, w tym inicjalizacja populacji, selekcja, naprawianie populacji, stworzenie następnej populacji itp.

\end{itemize}

W folderze \textbf{tests/} znajdują się testy jednostkowe, dla poszczególnych metod programu.

Do folderu \textbf{logs/} zapisywany jest wyniki eksperymentu opisanego w pliku \emph{Experiments.py}. Każdy plik logu zawiera przebieg ewolucji populacji (jakość najlepszą, oraz jakość uśrednioną), a także parametry wywołania instancji algorytmu. Nazwa logu opisuje parametry wywołania:
liczba elementów w plecaku, typ pojemności, korelacja, wybrany algorytm.

W folderze \textbf{img/} przechowywane są obrazki wykorzystane w dokumentacji. 

\pagebreak

\begin{thebibliography}{9}
\bibitem{knapsack} 
Zbigniew Michalewicz, Jarosław Arabas. "Genetic Algorithms for the 0/1 Knapsack Problem." 1994.
\end{thebibliography}

  
\end{document}
